\section{Сертификат открытого ключа}\label{FMT.Cert}

Сертификат открытого ключа описывается типом~\texttt{Certificate}, который 
определен в СТБ 34.101.19.

Сертификат составляется по правилам СТБ 34.101.19 со следующими уточнениями:
\begin{enumerate}

\item
Алгоритмы ЭЦП должны быть выбраны из перечня, заданного 
в~\ref{CRYPTO.Sign}.

\item
Серийный номер (компонент \texttt{serialNumber}) должен быть 
положительным целым числом, 
DER-код которого укладывается в 20 октетов.
%
УЦ должен гарантировать уникальность серийных 
номеров всех выпускаемых сертификатов, даже тех, которые 
подписываются на разных личных ключах УЦ.

\item
Идентификационные данные эмитента и субъекта (компоненты~\texttt{issuer} и~\texttt{subject} соответственно) заполняются в соответствии с подразделом~\ref{ENTITIES.Name}.

\item
Срок действия сертификата (компонент~\texttt{validity}) назначается 
эмитентом при выполнении 
процессов~\texttt{Enroll}, \texttt{Reenroll} и~\texttt{Spawn} 
(см.~\ref{PROCESSES.Enroll}, \ref{PROCESSES.Reenroll} 
и~\ref{PROCESSES.Spawn}).

Продолжительность действия сертификатов субъектов различных ролей
не должна превышать значений, указанных в таблице~\ref{Table.CERT.Validity}. 
%
При этом, как обычно, сертификаты операторов подчиняются правилам для ЮП, 
а сертификаты агентов~--- правилам для КА.

В процессе~\texttt{Reenroll} в новый сертификат может переноситься открытый 
ключ из старого. При таком продлении сертификата ограничения
таблицы~\ref{Table.CERT.Validity} не должны нарушаться.

\begin{table}
\caption{Сроки действия сертификатов}
\label{Table.CERT.Validity}
\begin{tabular}{|l|c|c|}
\hline
Сторона & Уровень стойкости & Максимальный срок\\
        &                   & действия (лет)\\
\hline
\hline

КУЦ & $l=128$ & 20\\
\cline{2-3} & $l=192$ & 30\\
\cline{2-3} & $l=256$ & 40\\
\hline

РУЦ & $l=128$ & 15\\
\cline{2-3} & $l=192$ & 20\\
\cline{2-3} & $l=256$ & 30\\
\hline

ПУЦ, СШВ,    & $l=128$ & 5\\
\cline{2-3}
СЗД, ЦАС,    & $l=192$ & 8\\
\cline{2-3} 
РЦ           & $l=256$ & 10\\
\hline

OCSP, TLS,  & $l=128$ & 3 \\
\cline{2-3}
СИ, КА & $l=192$ & 4\\
\cline{2-3} & $l=256$ & 5\\
\hline

ФЛ, ЮП & $l=128$ & 2 \\
\cline{2-3} & $l=192$ & 3 \\
\cline{2-3} & $l=256$ & 4 \\
\hline
\end{tabular}
\end{table}

\item
Правила описания открытого ключа (компонент~\texttt{subjectPublicKeyInfo})
определены в~\ref{CRYPTO.Pubkey}.

\item
Опциональный компонент \texttt{еxtensions} должен быть включен 
в сертификат. При его заполнении должны использоваться правила, 
определенные в~\ref{FMT.Ext}. 
\end{enumerate}

Все опциональные компоненты, про которые не сказано в уточнениях, 
должны быть опущены.