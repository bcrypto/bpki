\section{Процесс \texttt{Reenroll}}\label{PROCESSES.Reenroll}

Процесс~\texttt{Reenroll} выполняют субъект сертификата и УЦ~--- его 
эмитент. С помощью \texttt{Reenroll} субъект продлевает 
срок действия сертификата или изменяет в нем дополнительные
идентификационные атрибуты (см.~\ref{ENTITIES.SAN}).
%
При успешном завершении~\texttt{Reenroll} субъект получает новый 
сертификат, а его действующий сертификат отзывается. 

В новый сертификат может быть перенесен открытый ключ
действующего. При переносе сохраняется личный ключ,
и субъект избавляется от необходимости менять ключевой
контейнер или переписывать критическую память аппаратного КТ.

Процесс~\texttt{Reenroll} включает те же процедуры, 
что и процесс~\texttt{Enroll}. Исключается только процедура 
аутентификации. Процедуры~\texttt{Reenroll} в основном повторяют 
процедуры~\texttt{Enroll}. Отличия описываются ниже.

Генерация ключей выполняется также, как в сценарии~\texttt{Enroll1}. 
Если субъект переносит в новый сертификат открытый ключ действующего, то 
генерация не выполняется. 

Запрос на выпуск сертификата готовится также, как в~\texttt{Enroll1}.
%
Субъект указывает в запросе либо только что сгенерированный открытый ключ,
либо открытый ключ действующего сертификата.
%
Субъект переносит в запрос идентификационные данные из своего действующего 
сертификата. При переносе субъект может изменить только дополнительные 
идентификационные атрибуты. 

В атрибуте~\texttt{challengePassword} запроса субъект 
может обновить свой пароль отзыва и передать УЦ дополнительную
информацию (например, об оплате услуг доверия).

Субъект сам заверяет свой запрос, подписывая его на открытом ключе 
действующего сертификата. Подписанный запрос оформляется как 
контейнер~\texttt{SignedData}, конвертуется на открытом ключе~УЦ
и отправляется УЦ.

УЦ обрабатывает запрос по алгоритму процесса~\texttt{Enroll}
с учетом следующих корректировок.
\begin{enumerate}
\item
На шаге 3.1 алгоритма не проверять, что в
расширении~\texttt{CertificatePolicies} сертификата оправителя установлена
роль РЦ. Вместо этого проверить, что основные идентификационные атрибуты
в сертификате отправителя совпадают с основными идентификационными
атрибутами в запросе. Основные идентификационные атрибуты определены
в~\ref{ENTITIES.Attrs}.

\item
Если в запросе указаны новые дополнительные идентификационные атрибуты,
то на шаге 3.1 дополнительно \doubt{проверить}, что субъект владеет 
соответствующими цифровыми ресурсами. 

\item
Пропустить шаг~4.

\item
Если в запросе повторяется открытый ключ сертификата отправителя,
то на шаге 5.4 повторить начало действия этого сертификата в новом 
сертификате. Повтор означает накопление срока действия открытого ключа
при его переносе из сертификата в сертификат.
Отменить выпуск нового сертификата, если накопленный срок действия 
превышает границы, указанные в таблице~\ref{Table.CERT.Validity}.

\item
Выполнить дополнительный шаг: отозвать сертификат отправителя запроса.
\end{enumerate}

По результатам обработки запроса УЦ формирует ответ: новый сертификат или 
контейнер типа~\texttt{BPKIResp}. Новый сертификат конвертуется на открытом 
ключе самого сертификата. Контейнер \texttt{BPKIResp} описывает ошибку во 
время обработки запроса. Контейнер интерпретируется также, как в 
процессе~\texttt{Enroll}. Как и в~\texttt{Enroll} контейнер подписывается 
на ключе УЦ.

Ответ УЦ получает субъект. Получатель обрабатывает ответ также, 
как в процессе~\texttt{Enroll}. 
%
Если получено сообщение об ошибке со статусом
\texttt{waiting}, то субъект должен сохранить идентификатор запроса
и уточнить статус позже, выполнив процесс~\texttt{Retrieve}.

УЦ должен гарантировать действительность старого сертификата
в период сохранения статуса~\texttt{waiting}.

