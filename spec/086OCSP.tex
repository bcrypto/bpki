\section{Запрос и ответ OCSP}\label{FMT.OCSP}

\subsection{Формат запроса}

Формат запроса OCSP задается типом \texttt{OCSPRequest}, который определен 
в СТБ 34.101.26. В самом контейнере~\texttt{OCSPRequest} и во вложенных в него
контейнерах должны быть опущены все опциональные компоненты и компоненты 
со значениями по умолчанию.

Основная информационная часть~\texttt{OCSPRequest}~--- это список ссылок
на сертификаты, статус которых необходимо проверить.
%
Каждая ссылка описывается контейнером~\texttt{Request}, который заполняется
по правилам СТБ 34.101.26. Идентификатор алгоритма хэширования, указанный в
компоненте~\texttt{hashAlgorithm} контейнера~\texttt{Request}, должен
выбираться из перечня, заданного в~\ref{CRYPTO.Hash}.

\subsection{Формат ответа}

Формат ответа OCSP задается типом~\texttt{OCSPResponse}, который определен 
в СТБ 34.101.26. 

Контейнер~\texttt{OCSPResponse} заполняется по правилам СТБ 34.101.26
со следующими уточнениями.

\begin{enumerate}
\item
В компоненте~\texttt{signatureAlgorithm} 
контейнера~\texttt{BasicOCSPResponse}, вложенного   
в~\texttt{OCSPResponse}, должен быть указан алгоритм ЭЦП
из перечня, заданного в~\ref{CRYPTO.Sign}.

\item
Компонент~\texttt{certs} контейнера~\texttt{BasicOCSPResponse}
должен быть опущен либо в этом компоненте должен быть указан 
единственный сертификат~--- сертификат отправителя OCSP-ответа.

\item
В контейнере~\texttt{ResponseData}, вложенном 
в~\texttt{BasicOCSPResponse}, должны быть опущены 
компонент~\texttt{version} со значением по умолчанию
и опциональный компонент~\texttt{responseExtensions}.

\item
В каждом из контейнеров~\texttt{SingleResponse}, вложенных
в~\texttt{ResponseData}, должен быть опущен 
компонент~\texttt{singleExtensions}.
\end{enumerate}
