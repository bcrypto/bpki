\section{Запрос на отзыв сертификата}\label{FMT.BPKIRevokeReq}

Формат запроса на отзыв сертификата определяется следующим типом АСН.1:
\begin{verbatim}
BPKIRevokeReq ::= SEQUENCE {
  subject         Name,
  serialNumber    INTEGER,
  reason          CRLReason,
  revokePwd       UTF8String,
  comment         UTF8String }
\end{verbatim}

Компонент \texttt{subjectName} содержит идентификационные 
данные стороны, чей сертификат требуется отозвать. 

Компонент \texttt{serialNumber} содержит серийный номер 
сертификата, который требуется отозвать.

Компонент \texttt{reason} содержит причину отзыва сертификата. 
Тип \texttt{CRLReason} определен в СТБ 34.101.19.

Компонент \texttt{revokePwd} содержит пароль на отзыв 
сертификата, который является действительным в 
настоящий момент. Этот пароль может быть предварительно
задан в атрибуте~\texttt{challengePassword} запроса на получение 
сертификата (см. п.~\ref{FMT.CSR.CP}) или установлен с помощью 
процесса~\texttt{Chpwd} (см. подраздел~\ref{PROCESSES.Chpwd}).

Запрос на отзыв может подписываться субъектом сертификата,
и тогда компонент~\texttt{revokePwd} может принимать любое значение,
\addendum{в том числе быть пустой строкой.}

Компонент~\texttt{comment} содержит дополнительную информацию 
о причине отзыва.
 
\section{Список отозванных сертификатов}\label{FMT.CRL}

Список отозванных сертификатов содержит идентификационные данные 
издателя, время выпуска данного и следующего СОС, набор 
записей отозванных сертификатах, некоторые расширения и ЭЦП, 
которая заверяет вышеуказанную информацию. СОС описывается типом 
\texttt{CertificateList}, который определен в СТБ 34.101.19.
 
Идентификационные данные издателя (подразумевается УЦ, 
выпустивший СОС) задаются компонентом \texttt{issuer}, 
который должен повторять компонент \texttt{subject} из 
сертификата эмитента данного СОС.  

Время выпуска данного и следующего СОС указывается в 
компонентах \texttt{thisUpdate} и \texttt{nextUpdate}, 
соответственно.

Набор записей отозванных сертификатов указываются в компоненте 
\texttt{revokedCertificates}. При отсутствии отозванных  
сертификатов данный компонент не должен включаться в сертификат.
Каждая запись содержит серийный номер отозванного 
сертификата, дату и время отзыва. Помимо этого
каждая запись может содержать расширения, которые содержат
набор дополнительных атрибутов, связанных с данной записью. 
Данные расширения указываются в компоненте 
\texttt{crlEntryExtensions} типа \texttt{Extensions}. Как 
и расширения сертификата, они могут быть обязательными или необязательными,
критическими или некритическими. Расширение \texttt{reasonCode}
является обязательным, некритическим и содержит код причины 
отзыва сертификата.  

Расширения СОС описывается компонентом \texttt{crlExtensions}
типа~\texttt{CertificateList}. Должны использоваться следующие расширения: 
\begin{enumerate}
\item 
Некритическое расширение \texttt{AuthorityKeyIdentifier}. 
В расширении задается хэш-значение открытого ключа эмитента СОС.
Сохраняются заданные в п.~\ref{FMT.Ext.SKID} правила формирования 
расширения.

\item
Некритическое расширение \texttt{CRLNumber}. 
В расширении задается номер текущего СОС.
Номер представляется неотрицательным целым числом, 
\addendum{DER-код} которого укладывается в 20 октетов. 
Номера последовательных СОС должны монотонно возрастать.
\end{enumerate}
