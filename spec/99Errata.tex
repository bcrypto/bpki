\clearpage
\chapter*{\mbox{}\hfill Поправка к официальной редакции\hfill\mbox{}}

\mbox{}

{\small
\begin{center}
\begin{tabular}{|p{3cm}|p{6.3cm}|p{6.4cm}|}
\hline
В каком месте & Напечатано & Должно быть\\
\hline
\hline
Подраздел~\ref{COMMON.PKI},\par 
абзац 3
&
В конце концов получается цепочка сертификатов, в которой каждый следующий 
сертификат подписывается на открытом ключе текущего.
&
В конце концов получается цепочка сертификатов, в которой каждый следующий 
сертификат подписывается на личном ключе текущего.
\\
\hline
Подраздел~\ref{COMMON.CT},\par 
абзац 2
&
\ldots представляет собой файл-контейнер с защищенным личным 
ключом и сопутствующими программами\ldots
&
\ldots представляет собой файл-контейнер с защищенным личным 
ключом и сопутствующие программы\ldots
\\
\hline
Подраздел~\ref{CRYPTO.AlgId},\par 
абзац 1
&
Компонентами типа является идентификатор алгоритма (или пары 
связанных алгоритмов) и соответствующие параметры. 
&
Компонентами типа являются идентификатор алгоритма (или пары 
связанных алгоритмов) и соответствующие параметры. 
\\
\hline
Пункт~\ref{FMT.Ext.EKU},\par 
абзац 2
&
Расширение ExtKeyUsage не должно включаться в сертификаты УЦ, ЦАС, РЦ 
и должно включаться в сертификаты остальных сторон.
&
Расширение ExtKeyUsage не должно включаться в сертификаты УЦ, ЦАС и РЦ,
может включаться в сертификаты КА и должно включаться в сертификаты остальных 
сторон.
\\
\hline
Подраздел~\ref{FMT.SignedData},\par пункт~3,\par 
последний абзац 
&
Первый идентификатор определен в СТБ 34.101.81, второй~--- в СТБ 
34.101.82, остальные~--- в приложении~\ref{ASN1}.
&
Первый идентификатор определен в СТБ 34.101.82, второй~--- в СТБ 
34.101.81, остальные~--- в приложении~\ref{ASN1}.
\\
\hline
Подраздел~\ref{FMT.EnvelopedData},\par 
пункт 1
&
Версия синтаксиса (компонент~\texttt{version}) должна равняться~$2$. 
&
Версия синтаксиса (компонент~\texttt{version}) должна равняться~$0$. 
\\
\hline
Подраздел~\ref{FMT.EnvelopedData},\par 
пункт 5.1
&
версия применяемого синтаксиса (компонент~\texttt{version}) должна 
равняться~$2$;
&
версия применяемого синтаксиса (компонент~\texttt{version}) должна 
равняться~$0$;
\\
\hline
Пункт~\ref{FMT.TSP.Resp},\par
абзац 2 
&
\texttt{statusInfo} (3 раза)
&
\texttt{statusString} (3 раза)
\\
\hline
Подраздел~\ref{PROCESSES.Reenroll},\par 
абзац 7
&
Субъект сам заверяет свой запрос, подписывая его на открытом ключе 
действующего сертификата.
&
Субъект сам заверяет свой запрос, подписывая его на личном ключе 
действующего сертификата.
\\
\hline
\end{tabular}
\end{center}
}