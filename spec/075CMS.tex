\section{Подписанные данные}\label{FMT.SignedData}

Формат подписанных данных задается типом \texttt{SignedData}, 
который определен в СТБ 34.101.23. Тип описывает версию применяемого 
синтаксиса, список алгоритмов хэширования, подписываемые данные, 
сертификат подписывающей стороны и информацию о подписывающей стороне. 

Версия синтаксиса указывается в компоненте \texttt{version}.
Должна использоваться версия~$1$.

Список алгоритмов хэширования указывается в компоненте
\texttt{digestAlgorithms}. Список должен содержать единственный алгоритм,
и этот алгоритм должен быть выбран из перечня, 
заданного в~\ref{CRYPTO.Hash}.

Подписываемые данные указываются в компоненте~\texttt{encapContentInfo}. 
%
В зависимости от типа подписываемых данных, вложенный 
в~\texttt{encapContentInfo} компонент~\texttt{eContentType} 
должен принимать одно из следующих значений:
\begin{enumerate}
\item[1)]
\texttt{id-ct-TSTInfo}, если подписывается ответ СШВ;
\item[2)]
\texttt{id-ct-DVCSResponseData}, если подписывается ответ СЗД. 
\item[3)]
\texttt{id-data}, если подписываются неструктурированные данные:
запрос на получение сертификата, запрос на отзыв сертификата, ответ УЦ.
\end{enumerate}

% todo: сказать про подписанные атрибуты:
% - обязательно присутствуют для id-ct-TSTInfo, id-ct-DVCSResponseData
% - могут отсутствовать для id-data

% todo: нужен ли атрибут SigningTime?

Опциональный компонент~\texttt{certificates} должен быть опущен в случае, 
если подписанные данные создает РУЦ. В противном случае 
компонент~\texttt{certificates} должен содержать только сертификат 
подписывающей стороны.

Информация о подписывающей стороне указывается в
компоненте~\texttt{signerInfos}. Компонент должен содержать единственный
элемент типа~\texttt{SignerInfo}. Этот элемент описывает версию синтаксиса,
идентификатор подписывающей стороны, алгоритм хэширования, алгоритм ЭЦП и
саму ЭЦП. Версия должна быть равна $1$. Идентификатор подписывающей стороны
должен иметь тип~\texttt{issuerAndSerialNumber}. В 
компоненте~\texttt{issuer} данного типа должен быть указан 
компонент~\texttt{subject} из сертификата эмитента, а в  
компоненте~\texttt{serialNumber}~--- серийный номер сертификата 
подписывающей стороны. 

\addendum{
Алгоритм ЭЦП должен быть выбран из перечня, указанного в~\ref{CRYPTO.Sign}. 
Вспомогательный для алгоритма ЭЦП алгоритм хэширования должен
совпадать с алгоритмом, указанным в компоненте~\texttt{digestAlgorithms}. 
Опциональный компонент~\texttt{crls} должен быть опущен.
}

\section{Конвертованные данные}\label{FMT.EnvelopedData}

Формат конвертованных данных задается типом~\texttt{EnvelopedData}, который
определен в СТБ 34.101.23. Тип описывает версию применяемого синтаксиса,
конвертованные данные и информацию об их получателе.

Версия применяемого синтаксиса указывается в компоненте
\texttt{version}. Должна использоваться версия~$2$.

Конвертованные данные указываются в компоненте~\texttt{encryptedContentInfo}.
%
Вложенный в~\texttt{encryptedContentInfo} компонент~\texttt{contentEncryptionAlgorithm}
описывает алгоритм шифрования конвертованных данных. Используемый алгоритм 
должен быть выбран из перечня, заданного в~\ref{CRYPTO.Encr}.

В зависимости от типа конвертуемых данных, вложенный 
в~\texttt{encryptedContentInfo} компонент~\texttt{eContentType} 
должен принимать одно из следующих значений:
\begin{enumerate}
\item[1)]
\texttt{id-signedData}, если конвертуются подписанные данные;
или ответ УЦ; 
\item[2)]
\texttt{id-data}, если конвертуются неструктурированные данные:
запрос на получение сертификата, сертификат, пароль отзыва, запрос на 
отзыв сертификата.
\end{enumerate}

% todo: сказать про атрибуты -- отсутствуют

% todo: сказать, что originatorInfo отсутствует

Информация о получателе указывается в компоненте~\texttt{recipientInfos}.
Компонент должен содержать единственный элемент 
типа~\texttt{RecipientInfo}, и в этом элементе должен быть выбран  
компонент~\texttt{ktri} типа~\texttt{KeyTransRecipientInfo}.  
Компонент~\texttt{ktri} описывает версию применяемого синтаксиса, 
идентификатор получателя, алгоритм транспорта ключа и токен ключа. 
Версия должна быть равна 2. Идентификатор получателя должен 
иметь тип~\texttt{subjectKeyIdentifier}. Должен использоваться алгоритм 
транспорта ключа~\texttt{bign-keytransport}, описанный 
в~\ref{CRYPTO.Transport}. 

% todo: идентификация получателя skid vs issuer_and_serial. 
% Почему выбран skid?
