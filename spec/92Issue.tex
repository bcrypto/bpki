\begin{appendix}{А}{рекомендуемое}{Выпуск облегченных сертификатов}
\label{ISSUE}

\mbox{}

На КТ и терминалы во время их персонализации записываются не обычные,
а облегченные сертификаты. Содержание и формат облегченных сертификатов 
описаны в СТБ 34.101.79.

За персонализацию отвечает специальный РЦ. Этот РЦ повторяет стандартные 
процедуры~\texttt{Enroll}. 

\texttt{Issue}. 
РЦ персонализирует аппаратный КТ субъекта. 
Аутентификация не проводится. РЦ самостоятельно генерирует ключи,
готовит запрос, заверяет и отправляет его, записывает сертификат из ответа 
в КТ.

Третий сценарий должен использоваться при массовом выпуске КТ.
\addendum{
С помощью \texttt{Issue} могут выпускаться как стандартные сертификаты, 
так и облегченные.
}

\hiddensection{Генерация ключей}\label{PROCESSES.Enroll.Gen}

В~\texttt{Issue} ключи генерирует РЦ. 
РЦ может использовать сгенерированный личный ключ только для 
подписи запроса на получение сертификата. РЦ должен уничтожить личный ключ 
после подписи запроса и записи ключа на КТ.
%
Вместе с ключом на КТ сохраняются идентификационные данные субъекта,
которые РЦ получает у специальной доверенной службы. 
Взаимодействие РЦ и службы в настоящем стандарте не рассматривается.

% todo: при выпуске облегченных сертификатов:
% - РЦ должен указать в запросе расширение с правами доступа.
% - В id-атрибуте serialNumber д.б. указан серийный номер КТ.
% - Передаваемые в запросе id-данные не попадут в сертификат (только sn).
%   Нужно ли передавать?
% - РЦ создает на КТ контейнер Name?

\hiddensection{Подготовка запроса}\label{PROCESSES.Enroll.CSR}

В~\texttt{Issue} запрос готовит~РЦ, который получает идентификационные 
данные субъекта у специальной службы. Взаимодействие РЦ со службой в 
настоящем стандарте не рассматривается.

% todo: как заполняется challengePassword?

В~\texttt{Issue} аутентификация не проводится.

\hiddensection{Заверение запроса}

В~\texttt{Enroll2}, \texttt{Issue} запрос заверяет РЦ.
РЦ получает идентификационные данные от доверенных источников,
сам готовит по ним запрос и поэтому никаких проверок перед заверением
не проводит. \doubt{В \texttt{Issue} РЦ может заверить, 
т.~е. включить в контейнер \texttt{SignedData}, 
не один, а сразу несколько запросов.} 

\hiddensection{Отправка запроса}\label{PROCESSES.Enroll.Enveloped}

В~\texttt{Issue} отправку выполняет РЦ.

\hiddensection{Обработка запроса}\label{PROCESSES.Enroll.Issue}

УЦ обрабатывает запрос также как запросы Enroll1--Enroll2.
Только выпускает не обычный, а облегченный сертификат.

\doubt{
В \texttt{Issue} при выпуске облегченного сертификата 
идентификационные данные не переносятся в сертификат целиком.
Облегченный сертификат кроме открытого ключа и подписи УЦ содержит 
серийный номер КТ и специальное расширение с правами доступа.
}

\hiddensection{Обработка ответа}

Если в контейнере~\texttt{BPKIResp} указан статус~\texttt{waiting}, то сертификат 
все-таки может быть получен через повторное обращение к УЦ с помощью 
процесса~\texttt{Retrieve}. В обращении должен использоваться 
идентификатор запроса. 
%
В~\texttt{Issue} РЦ не передает идентификатор субъекту, 
а самостоятельно выполняет процесс~\texttt{Retrieve}.

\end{appendix}
  