\section{Расширения сертификата открытого ключа}\label{FMT.Ext}

\subsection{Общие положения}\label{FMT.Ext.Intro} 

Расширения сертификата описывают дополнительную информацию о субъекте,
эмитенте или о самом сертификате. В зависимости от роли владельца 
сертификат может содержать те или иные наборы расширений.

В соответствии с СТБ 34.101.19  критическим или некритическим. Также Каждое расширение 
можно классифицировать как обязательное или необязательное. Обязательное 
расширение~--- это расширение, которое должно присутствовать в 
сертификате.  Критическое расширение~--- это обязательное 
расширение, которое должно быть корректным: при нарушении корректности
сторона, проверяющая сертификат, должна завершить проверку с ошибкой.

Далее определяются допустимые расширения сертификатов, издаваемых КУЦ, РУЦ и ПУЦ. 
Если не оговорено противное, каждое из определяемых расширений является 
обязательным для каждой роли владельца сертификата. 

\addendum{Сертификаты, издаваемые ПУЦ, могут содержать расширения, 
дополнительные к перечисляемым ниже. Включение дополнительных расширений в 
сертификаты, издаваемые КУЦ и РУЦ, запрещено.}

\subsection{Расширения \texttt{SubjectKeyIdentifier} и 
\texttt{AuthorityKeyIdentifier}}\label{FMT.Ext.SKID} 

Расширения \texttt{SubjectKeyIdentifier} и \texttt{authorityKeyIdentifier} 
описывают хэш-значения открытых ключей эмитента и субъекта сертификата 
соответственно. 

Расширения являются обязательными и некритическими.
 
Хэш-значения должны вычисляться либо с помощью алгоритма~\texttt{belt-hash}, 
определенного в СТБ 34.101.31, либо с помощью алгоритма~SHA-1,
определенного в~\cite{SHA1}. В первом случае хэш-значение 
представляет собой строку из~$32$ октетов, во втором~--- из~$20$ октетов.

\begin{note}
Примечание~--- 
Расширения~\texttt{SubjectKeyIdentifier} и \texttt{AuthorityKeyIdentifier}
облегчают управление сертификатами, не решая при этом криптографических
задач. Поэтому в расширениях разрешается использовать алгоритм хэширования
SHA-1, признанный на сегодняшний день криптографически нестойким.
Разрешение на использование SHA-1  продиктовано необходимостью 
обеспечивать совместимость с действующими системами защиты информации. 
В тех случаях, когда совместимость не нужна, следует 
использовать~\texttt{belt-hash}. 
\end{note}

\subsection{Расширение \texttt{KeyUsage}}\label{FMT.Ext.KU}

Расширение~\texttt{KeyUsage} описывает назначение открытого ключа. 
Описание представляет собой комбинацию флагов, определенных в СТБ 34.101.19.

Расширение является критическим.

В таблице~\ref{Table.Fmt.keyUsage} перечислены флаги \texttt{KeyUsage} 
для сторон различных ролей. 
%
Пропуск в таблице означает обязательное отсутствие флага,
плюc~--- обязательное присутствие.

\begin{table}
\caption{Флаги \texttt{KeyUsage}}
\label{Table.Fmt.keyUsage}
\begin{tabular}{|l|c|c|c|c|c|}
\hline
Сторона & 
\rotatebox{90}{\texttt{digitalSignature}~} &
\rotatebox{90}{\texttt{nonRepudiation}~} & 
\rotatebox{90}{\texttt{keyEncipherment}~} & 
\rotatebox{90}{\texttt{keyCertSign}~} & 
\rotatebox{90}{\texttt{cRLSign}~}\\
\hline
\hline
КУЦ        & & & &  + & + \\
\hline
РУЦ        & + & &  + & + & + \\
\hline
ПУЦ		   & + & &  + & + & + \\
\hline
ЦАС		   &  & & & + & + \\
\hline
РЦ		   & + & + & + &  & \\
\hline
OCSP-сервер & + & + &  & & \\
\hline
СШВ        & + & + &  & & \\
\hline
СЗД        & + & + &  & & \\
\hline
СИ		   & + & + & + & & \\
\hline
TLS-сервер & + &  &  + & & \\
\hline
ФЛ    	   & + & + & +  & & \\
\hline
ЮП 		   & + & + & + &  & \\
\hline
КА         & + & + & + & & \\
\hline                                     
\end{tabular}
\end{table}

% todo: В СТБ 34.101.67 дан пример сертификата ЦАС. В сертификате 
% установлены (?) флаги digitalSignature и cRLSign. Разобраться.
% reply: Никаких ограничений на установление каких-либо флагов я не нашел. 
% Кроме того, что если установлен keyCertSign, то в basicConstaints 
% указывается CA = true. Дальше отвечу по поводу digitalSignature.

% todo: найден сертификат ЦАС, в расширении basicConstaints которого 
% CA = true. Разобраться.
% reply: Цитата из rfc 5755: So, the AC issuer's PKC MUST NOT have a
% basicConstraints extension with the cA boolean set to TRUE. Эта 
% цитата влечет за собой снятие флага keyCertSign, и установку 
% флага в digitalSignature. Такой вариант выглядит наиболее разумным.
% К тому же, будет соответствие с примером из СТБ.

\subsection{Расширение \texttt{ExtKeyUsageSyntax}}\label{FMT.Ext.EKU}

Расширение \texttt{ExtKeyUsageSyntax} описывает область применения ключей 
сертификата. Описание представляет собой набор идентификаторов АСН.1. 

Расширение \texttt{ExtKeyUsageSyntax} не должно включаться в сертификаты КУЦ
и должно включаться в сертификаты остальных сторон.
Расширение является критическим.

В сертификате OCSP-сервера расширение должно содержать
идентификатор \verb|id-kp-OCSPSigning|, определенный в СТБ 34.101.19.

В сертификате СЗД расширение должно содержать
идентификатор \verb|id-kp-dvcs|, определенный в СТБ 34.101.dvcs.

В сертификате СШВ расширение должно содержать
идентификатор \verb|id-kp-timeStamping|, определенный в СТБ 34.101.19.

В сертификатах СИ и TLS-сервера расширение должно содержать
идентификатор \verb|id-kp-serverAuth|, определенный в СТБ 34.101.19.

В сертификатах ФЛ и ЮП расширение должно содержать
идентификатор \verb|id-kp-clientAuth|, определенный в СТБ 34.101.19.

В сертификате стороны, которая выступает в роли сервера (клиента) 
терминального режима, расширение \texttt{ExtKeyUsageSyntax} должно содержать 
идентификатор \verb|bpki-eku-serverTM|  
(\verb|bpki-eku-clientTM|). Идентификаторы определены в приложении~\ref{ASN1}.

% todo: emailProtection для SignedData / EnvelopedData?
% reply: Как я понял, данный флаг используется если генерировать подпись в
% Mime-формате. Подпись в openssl без этого флага генерируется.

\subsection{Расширение \texttt{certificatePolicies}}\label{FMT.Ext.CP}

Расширение \texttt{certificatePolicies} описывают политику, в соответствии 
с которой был выпущен сертификат, и цели, в которых сертификат может 
использоваться. 

Расширение~\texttt{certificatePolicies} не должно включаться в сертификаты
КУЦ и должно включаться в сертификаты остальных сторон. 
Расширение является \addendum{некритическим}.

Описание политики состоит из пунктов, представленных идентификаторами 
АСН.1. В пунктах должны быть опущены опциональные классификаторы. 

В сертификатах РУЦ и ПУЦ расширение должно содержать единственный пункт
с идентификатором~\texttt{anyPolicy}, определенным в СТБ 34.101.19.

В сертификате конечного участника расширение 
должно содержать пункты с идентификаторами его ролей.
Идентификаторы определены в приложении~\ref{ASN1}
в соответствии с таблицей~\ref{Table.ENTITIES.Roles}. 
Расширение может включать пункты нескольких ролей.
Например, в сертификате оператора РЦ указываются две роли: ЮП и РЦ.

В сертификате конечного участника расширение~\texttt{certificatePolicies} 
может содержать дополнительные пункты, отличные от пунктов 
ролей. Например, пункт политики, в соответствии с которой 
проверялось владение TLS-сервером заявленным DNS-именем.

\subsection{Расширение \texttt{BasicConstraints}}

Расширение~\texttt{BasicConstraints} дифференцирует сертификаты УЦ и
конечных участников. Дополнительно расширение ограничивает 
длину цепочек сертификатов, подчиненных сертификату УЦ.

Расширение \texttt{BasicConstraints} является критическим.

В сертификатах УЦ флаг~\texttt{сA} расширения должен быть 
установлен, в сертификатах конечных участников~--- сброшен. 

В сертификатах КУЦ, РУЦ и конечных участников компонент 
\texttt{pathLenConstraint} должен отсутствовать,
а в сертификате ПУЦ~--- принимать значение~0. 
Это значение означает запрет на выпуск ПУЦ сертификатов 
другим УЦ. 

\subsection{Расширение \texttt{SubjectAltName}}

Расширение~\texttt{SubjectAltName} содержит дополнительные 
идентификационные данные, описанные в подразделе~\ref{ENTITIES.SAN}. 

Расширение является некритическим и необязательным.

\subsection{Расширение \texttt{CRLDistributionPoints}}

Расширение \texttt{CRLDistributionPoints} описывает расположение СОС, 
выпускаемых эмитентом сертификата. 

Расширение~\texttt{CRLDistributionPoints} 
не должно включаться в сертификаты КУЦ и должно включаться в
сертификаты остальных сторон. Расширение является некритическим и обязательным.

Отдельные точки распространения списков отзыва описываются 
типом~\texttt{DistributionPoint}. Опциональные компоненты \texttt{reasons} 
и \texttt{cRLIssuer} этого типа должны быть опущены, а в 
компоненте~\texttt{distributionPoint} должен быть указан 
URI-адрес точки распространения (через выбор сначала 
варианта~\texttt{fullName}, а затем 
варианта~\texttt{uniformResourceIdentifier}). 

\subsection{Расширение \texttt{AuthorityInfoAccessSyntax}}

Расширение \texttt{AuthorityInfoAccessSyntax} описывает расположение 
OCSP-сервера, к которому следует обратиться, чтобы получить информацию о 
статусе сертификата.

Расширение не должно включаться в сертификаты КУЦ и РУЦ и должно 
включаться в сертификаты остальных сторон. Расширение является 
некритическим. 

Расширение должно включать единственный компонент 
типа~\texttt{AccessDescription}. Каждый такой компонент, в свою очередь, 
состоит из компонентов \texttt{accessMethod} и \texttt{accessLocation}. 
%
Во вложенном в него компоненте \texttt{accessMethod} должен быть установлен 
идентификатор~\verb|id-ad-ocsp|, определенный в СТБ 34.101.19,
%
а во вложенном компоненте~\texttt{accessLocation}~--- URI-адрес OCSP-сервера
(через выбор варианта~\texttt{uniformResourceIdentifier}).

