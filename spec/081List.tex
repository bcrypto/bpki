\section{Перечень процессов}\label{PROCESSES.List}

Управление сертификатами конечных участников ИОК реализуется через 
следующие процессы. 

\begin{enumerate}
\item
\texttt{Enroll}~--- 
выпуск сертификата для стороны, которая располагает действительным
удостоверением, но не обязательно действительным сертификатом.
\item
\texttt{Reenroll}~--- 
обновление действительного сертификата.
\item
\texttt{Spawn}~--- 
выпуск нового сертификата для стороны, которая располагает действительным
сертификатом.
\item
\texttt{Retrieve}~--- 
получение сертификата, который был запрошен в процессах~\texttt{Enroll},
\texttt{Reenroll}, \texttt{Spawn} и выпуск которого задерживается.
\item
\texttt{Chpwd}~--- 
установка (изменение) пароля отзыва сертификата.
\item
\texttt{Revoke}~--- 
отзыв сертификата.
\end{enumerate}

Каждый процесс представляет собой последовательность определенных процедур. 
При ошибке в любой из процедур процесс также завершается с ошибкой.

Процесс может включать процедуры отправки запроса УЦ и получения 
соответствующего ответа. Форматы запросов и ответов схематически представлены 
в таблице~\ref{Table.PROCESSES.Fmt}. Используются имена типов данных и их 
компонентов, описанные в разделе~\ref{FMT}. Нижний индекс~C означает 
субъекта сертификата, нижний индекс~опРЦ~--- оператора~РЦ.
Нижний индекс \texttt{SignedData} указывает на подписанта,
нижний индекс \texttt{EnvelopedData}~--- на получателя конвертованных 
данных.

\begin{table}[bht]
\caption{Схемы форматов запросов~/ ответов}
\label{Table.PROCESSES.Fmt}
\begin{tabular}{|l|l|}
\hline
\multicolumn{2}{|c|}{Процесс \texttt{Enroll}}\\
\hline
\hline
\rule{0pt}{18pt}
Запрос &
$\texttt{EnvelopedData}_{\text{УЦ}}\bigl(
\texttt{SignedData}_{\text{РЦ~| опРЦ}}
(\texttt{CertificateRequest}_{\text{С}})\bigr)$ или\\
&
$\texttt{EnvelopedData}_{\text{УЦ}}(
\texttt{CertificateRequest}_{\text{С}}[\texttt{enrollPwd}])$\\[6pt]
\hline                                      
%
\rule{0pt}{18pt}
Ответ &
$\texttt{EnvelopedData}_{\text{РЦ~| опРЦ~| С}}(\texttt{Certificate}_{\text{С}})$ или\\
&
$\texttt{SignedData}_{\text{УЦ}}(\texttt{BPKIResp})$\\[6pt]
\hline                                     
\hline
\multicolumn{2}{|c|}{Процессы \texttt{Reenroll} и \texttt{Spawn}}\\
\hline
\hline
\rule{0pt}{15pt}
Запрос &
$\texttt{EnvelopedData}_{\text{УЦ}}\Bigl(
\texttt{SignedData}_{\text{С}}
(\texttt{CertificateRequest}_{\text{С}})\Bigr)$\\[3pt]
\hline                                      
%
\rule{0pt}{15pt}
Ответ &
$\texttt{EnvelopedData}_{\text{С}}(\texttt{Certificate}_{\text{С}})$ или\\
&
$\texttt{SignedData}_{\text{УЦ}}(\texttt{BPKIResp})$\\[3pt]
\hline                                     
\hline
\multicolumn{2}{|c|}{Процесс \texttt{Retrieve}}\\
\hline
\hline
\rule{0pt}{15pt}
Запрос &
\texttt{BPKIRetrieveReq}\\[3pt]
\hline                                      
%
\rule{0pt}{15pt}
Ответ &
$\texttt{EnvelopedData}_{\text{РЦ~| опРЦ~| С}}(\texttt{Certificate}_{\text{С}})$ или\\
&
$\texttt{SignedData}_{\text{УЦ}}(\texttt{BPKIResp})$\\[3pt]
\hline                                     
\hline
\multicolumn{2}{|c|}{Процесс \texttt{Chpwd}}\\
\hline
\hline
\rule{0pt}{15pt}
Запрос &
$\texttt{EnvelopedData}_{\text{УЦ}}(\texttt{revokePwd})$\\[3pt]
\hline                                      
%
\rule{0pt}{15pt}
Ответ &
$\texttt{SignedData}_{\text{УЦ}}(\texttt{BPKIResp})$\\[3pt]
\hline                                     
\hline
\multicolumn{2}{|c|}{Процесс \texttt{Revoke}}\\
\hline
\hline
\rule{0pt}{15pt}
Запрос &
$\texttt{EnvelopedData}_{\text{УЦ}}
\bigl(\texttt{SignedData}_{\text{С}}(\texttt{BPKIRevoke})\bigr)$ или\\
&
$\texttt{EnvelopedData}_{\text{УЦ}}(\texttt{BPKIRevoke})$\\[3pt]
\hline                                      
%
\rule{0pt}{15pt}
Ответ &
$\texttt{SignedData}_{\text{УЦ}}(\texttt{BPKIResp})$\\[3pt]
\hline                                     
\end{tabular}
\end{table}

Перечисленные в таблице запросы и ответы транспортируются в виде пакетов 
HTTP. Правила транспорта определяются в разделе~\ref{TRANSPORT}.

