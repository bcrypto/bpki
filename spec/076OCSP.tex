\section{Запрос и ответ OCSP}\label{FMT.OCSP}

\subsection{Формат запроса}

Формат запроса OCSP задается типом \texttt{OCSPRequest}, который определен 
в СТБ 34.101.26. В самом контейнере~\texttt{OCSPRequest} и во вложенных в него
контейнерах должен быть опущены все опциональные компоненты и компоненты 
со значениями по умолчанию.

Основная информационная часть~\texttt{OCSPRequest}~--- это список ссылок
на сертификаты, статус которых необходимо проверить.
%
Каждая ссылка описывается контейнером~\texttt{Request}, который заполняется
по правилам СТБ 34.101.26. Идентификатор алгоритма хэширования, указанный в
компоненте~\texttt{hashAlgorithm} контейнера~\texttt{Request}, должен
выбираться из перечня, заданного в~\ref{CRYPTO.Hash}.

\subsection{Формат ответа}

Формат ответа OCSP задается типом~\texttt{OCSPResponse}, который определен 
в СТБ 34.101.26.  Форматы ответов различаются в зависимости от статуса 
обработки запроса. При успешной обработке ответ содержит статус обработки 
запроса и информационную часть ответа. При ошибке~--- только статус обработки.

Статус обработки запроса указывается в компоненте~\texttt{responseStatus}.
%
В случае успеха~\texttt{responseStatus} должен принимать значение~\texttt{successful}.
%
Если при обработке запроса произошла ошибка, то 
компонент~\texttt{responseStatus} должен принимать одно из трех значений: 
\texttt{tryLater}~--- ошибка произошла по причине загруженности сервера, 
\texttt{malforedRequest}~--- неверный формат запроса, 
\texttt{internalError}~--- внутренняя ошибка сервера.

Информационная часть ответа состоит из типа ответа
(\texttt{responseType}) и значения ответа (\texttt{response}).
 
Компонент~\texttt{responseType} должен принимать значение
\texttt{id-pkix-ocsp-basic}, определенное в СТБ 34.101.26. 
Компонент~\texttt{response} содержит \addendum{DER-код значения} 
типа~\texttt{BasicOCSPResponse}. Тип~\texttt{BasicOCSPResponse} описывает 
ответы по всем запрашиваемым сертификатам, алгоритм подписи и значение  
ЭЦП. Опциональный компонент~\texttt{certs} должен быть опущен.

% todo: без сертификата OCSP-сервера?

Ответы по запрашиваемым сертификатам представлены типом 
\texttt{ResponseData}. Тип описывает версию применяемого синтаксиса, 
информацию об OCSP-сервере, время создания ответа (компонент 
\texttt{producedAt}) и список ответов по всем проверяемым сертификатам. 
Версия применяемого синтаксиса указывается в компоненте
\texttt{version}\addendum{,} значение которого должны быть $0$. 
Информация об OCSP-сервере описывается типом~\texttt{ResponderID}.
В этом типе должен быть выбран компонент~\texttt{byName},
который содержит компонент~\texttt{subject} 
сертификата OCSP-сервера. Компонент \texttt{responseExtensions} должен быть опущен.

% todo: жестко без нонсов?

Каждый ответ из списка проверяемых сертификатов представляет собой
компонент типа \texttt{SingleResponse}. Такие компоненты содержат
информацию об эмитенте сертификата, статус сертификата (компонент
\texttt{certStatus}) и временные рамки действия ответа (компоненты
\texttt{thisUpdate} и \texttt{nextUpdate}). Опциональный компонент
\texttt{singleExtensions} должен быть опущен. Информация об эмитенте
сертификата должна быть представлена компонентом~\texttt{reqCert} из
запроса.

% todo: extUpdate указывается?

В зависимости от результата обработки запроса, компонент 
\texttt{certStatus} должен принимать одно из следующих значений: 
\begin{enumerate}
\item \texttt{good}~--- в случае, если сертификат не отозван;
\item \texttt{revoked}~--- в случае, ели сертификат отозван;
\item \texttt{unknown}~--- в случае, если серверу ничего не известно про 
данный сертификат. 
\end{enumerate}
 
% todo: профилировать RevokedInfo

