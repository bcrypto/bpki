\chapter{Термины и определения}

В настоящем стандарте применяют термины, устaновленные в СТБ 34.101.19,
СТБ 34.101.23, СТБ 34.101.27, СТБ 34.101.45, СТБ 34.101.65, 
СТБ 34.101.67, СТБ 34.101.tsp, СТБ 34.101.dvcs, а также следующие термины с 
соответствующими определениями:

{\bf \thedefctr~аутентификация}:
Проверка подлинности стороны.
Состоит в подтверждении ее идентичности.

{\bf \thedefctr~защищенное соединение}:
Соединение, которое обеспечивает конфиденциальность, 
контроль целостности и возможно подлинности сообщений. 
%
Контроль подлинности сообщений от~$A$ к~$B$ появляется тогда,
когда~$B$ провела аутентификацию~$A$.

{\bf \thedefctr~идентификатор}:
Вид идентификационных данных, строка сравнительно небольшой длины,
которую удобно использовать в информационных системах. 

{\bf \thedefctr~идентификационный атрибут}:
Компонент идентификационных данных. 

{\bf \thedefctr~идентичность, идентификационные данные}:
Набор атрибутов стороны, которые однозначно ее характеризуют в 
определенном контексте. В разных контекстах могут использоваться 
различные идентификационные данные одной и той же стороны.

{\bf \thedefctr~имя}:
Вид идентификационных данных, строка сравнительно небольшой длины
в естественном алфавите, которой удобно управлять человеку.

{\bf \thedefctr~клиентская программа}:
Программа, которая организует взаимодействие между 
пользователем, его криптографическими токенами и другими 
(как правило, удаленными) сторонами.

{\bf\thedefctr~криптографический автомат}: 
Средство криптографической защиты информации, выступающее от 
\addendum{собственного} лица при взаимодействии с другими сторонами.
%
Основное время работает автономно, без внешнего управления.
%
Примерами автоматов являются IP-шифраторы, устройства Интернета вещей.

{\bf\thedefctr~криптографический токен}: 
Средство криптографической защиты информации, имеющее конкретного 
владельца и выступающее от его лица при взаимодействии с другими 
сторонами. 
%
В настоящем стандарте токен хранит один или несколько личных ключей 
владельца и реализует операции с ними.  
%
На токене могут размещаться идентификационные данные владельца. 

{\bf \thedefctr~конвертованные данные}:
Данные, защищенные на секретном ключе, 
защищенном в свою очередь на открытом ключе получателя. 

{\bf \thedefctr~подписанные данные}:
Данные, сопровождаемые электронной цифровой подписью отправителя. 

{\bf\thedefctr~пользователь}: 
\addendum{Физическое лицо.}

{\bf\thedefctr~поставщик услуг доверия}:
Сторона, которая помогает установить доверенные отношения между другими 
сторонами, предоставляя определенную информацию по их запросу.
К поставщикам относятся удостоверяющие центры, центры атрибутных 
сертификатов, регистрационные центры, OCSP-серверы, 
службы штампов времени, заверения данных и идентификации.

{\bf\thedefctr~прикладная система}:
Информационная система, которая использует услуги доверия,
предоставляемые инфраструктурой открытых ключей.
%
% Организует аутентификацию пользователя 
% и свою авторизацию на доступ к его ресурсам.
% Имеет собственные ресурсы-услуги, которые могут быть предоставлены 
% пользователям после их аутентификации. 

{\bf\thedefctr~регистрационный центр}: 
Поставщик услуг доверия, который формирует идентификационные данные 
пользователя или проверяет и заверяет их для удостоверяющего центра. 
%
Регистрационный центр может дополнительно регистрировать 
аутентификационные данные пользователя и передавать их службе 
идентификации.

{\bf\thedefctr~служба идентификации}: 
Поставщик услуг доверия, который проводит аутентификацию 
пользователей и авторизует на доступ к их ресурсам.
%
% авторизует \emph{прикладную систему} на доступ к ресурсам пользователей.

{\bf\thedefctr~сторона}: 
Активный элемент (сервер, служба, центр, лицо, устройство), являющийся 
частью инфраструктуры открытых ключей или использующий сервисы инфраструктуры. 
Как правило, располагает одним или несколькими сертификатами 
открытых ключей, выпущенными в инфраструктуре.

{\bf\thedefctr~терминал}: 
Сторона, которая взаимодействует с криптографическим токеном по 
защищенному соединению после аутентификации перед ним и, возможно, 
встречной аутентификации токена.  

{\bf\thedefctr~удостоверение}: 
Физический (бумага, пластик) документ,
выпущенный доверенной стороной и содержащий идентификационные данные 
пользователя или организации. 

{\bf\thedefctr~физическое лицо}: 
Гражданин Республики Беларусь (резидент) или другой страны (нерезидент),
субъект гражданского права. 

{\bf\thedefctr~цепочка сертификатов}: 
Маршрут сертификации.

{\bf\thedefctr~юридическое лицо}:
Зарегистрированная в Республике Беларусь организация, 
субъект гражданского права.

{\bf\thedefctr~юридический представитель}: 
Физическое лицо, сотрудник или представитель юридического лица.

