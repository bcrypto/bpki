\section{Повторный запрос}\label{FMT.BPKIRetrieveReq}

Если получен ответ УЦ со статусом~\texttt{waiting}, то этот ответ можно 
уточнить, обратившись к УЦ повторно. Формат повторного запроса 
определяется следующим типом АСН.1:
\begin{verbatim}
BPKIRetrieveReq ::= SEQUENCE { 
  requestId   OCTET STRING(SIZE(32)),
  nonce       OCTET STRING(SIZE(8))}
\end{verbatim}

Компонент~\texttt{requestId} содержит идентификатор первоначального 
запроса, компонент~\texttt{nonce}~--- синхропосылку. Синхропосылка 
выбирается случайно отправителем повторного запроса. 

При обработке повторного запроса~\texttt{BPKIRetrieveReq} УЦ
переносит его компоненты в свой ответ~\texttt{BPKIResp}.

