\section{Подписанные данные}\label{FMT.SignedData}

Формат подписанных данных задается типом \texttt{SignedData}, 
который определен в СТБ 34.101.23. 

Контейнер~\texttt{SignedData} заполняется по правилам СТБ~34.101.23
со следующими уточнениями.

\begin{enumerate}
\item
Версия синтаксиса (компонент \texttt{version}) должна равняться~$1$.

\item
Список идентификаторов алгоритмов хэширования (компонент 
\texttt{digestAlgorithms}) должен содержать единственный элемент, и этот 
элемент должен быть выбран из перечня, заданного в~\ref{CRYPTO.Hash}.

\item
Тип подписываемых данных (компонент~\texttt{eContentType}, вложенный 
в~\texttt{encapContentInfo}) должен принимать одно из следующих значений:

\begin{enumerate}
\item[1)]
\texttt{id-ct-TSTInfo}, если подписывается ответ СШВ (см.~\ref{FMT.TSP.Resp});
\item[2)]
\texttt{id-ct-DVCSResponseData}, если подписывается ответ СЗД 
(см.~\ref{FMT.DVCS.Resp});
\item[3)]
\texttt{bpki-ct-enroll1-req}, если в сценарии~\texttt{Enroll1} 
процесса~\texttt{Enroll} подписывается запрос на выпуск сертификата 
(см.~\ref{PROCESSES.Enroll.Signed}); 
\item[4)]
\texttt{bpki-ct-enroll2-req}, если в сценарии~\texttt{Enroll2} 
процесса~\texttt{Enroll} подписывается запрос на выпуск сертификата 
(см.~\ref{PROCESSES.Enroll.Signed}); 
\item[5)]
\texttt{bpki-ct-reenroll-req}, если в процессе~\texttt{Reenroll} 
подписывается запрос на выпуск сертификата 
(см.~\ref{PROCESSES.Reenroll}); 
\item[6)]
\texttt{bpki-ct-spawn-req}, если в процессе~\texttt{Spawn} 
подписывается запрос на выпуск сертификата 
(см.~\ref{PROCESSES.Spawn}); 
\item[7)]
\texttt{bpki-ct-setpwd-req}, если в процессе~\texttt{Setpwd} 
подписывается новый пароль (см.~\ref{PROCESSES.Setpwd}); 
\item[8)]
\texttt{bpki-ct-revoke-req}, если в процессе~\texttt{Revoke} 
подписывается запрос на отзыв сертификата (см.~\ref{PROCESSES.Revoke}); 
\item[9)]
\texttt{bpki-ct-resp}, если подписывается ответ УЦ 
(см.~\ref{FMT.BPKIResp}).
\end{enumerate}

Первый идентификатор определен в СТБ 34.101.81, второй~--- в СТБ 
34.101.82, остальные~--- в приложении~\ref{ASN1}.

\item
Опциональный компонент~\texttt{certificates} должен присутствовать и 
должен содержать единственный сертификат~--- сертификат подписанта.
%
Опциональный компонент~\texttt{crls} должен быть опушен.

\item
Список данных о подписантах (компонент~\texttt{signerInfos}) должен 
содержать единственный контейнер типа~\texttt{SignerInfo}, и этот 
контейнер должен быть заполнен следующим образом: 

\begin{enumerate}
\item[1)]
версия (компонент \texttt{version}) должна равняться~$1$;
\item[2)]
идентификатор подписанта (\texttt{sid}) должен быть задан через
тип~\texttt{IssuerAndSerialNumber}. Компоненты~\texttt{issuer} 
и~\texttt{serialNumber} этого типа должны повторять одноименные компоненты 
сертификата подписанта;
\item[3)] 
идентификатор алгоритма хэширования (\texttt{digestAlgorithm}) должен 
совпадать с идентификатором, указанным в 
компоненте~\texttt{digestAlgorithms} основного  
контейнера~\texttt{SignedData};
\item[4)]
идентификатор алгоритмов ЭЦП (\texttt{signatureAlgorithm}) должен 
быть выбран из перечня, заданного в~\ref{CRYPTO.Sign}. 
Алгоритмы ЭЦП должны соответствовать алгоритму хэширования
из~\texttt{digestAlgorithm} и открытому ключу подписанта;
\item[5)]
в список подписанных атрибутов (\texttt{signedAttrs}) должны 
быть включены атрибуты <<Тип содержимого>> (\texttt{ContentType}),
<<Хэш-значение>> (\texttt{MessageDigest}) и может быть включен
атрибут <<Время подписания>> (\texttt{SigningTime}). 
Все атрибуты определены в СТБ 34.101.23;
\item[6)]
список неподписанных атрибутов (\texttt{unsignedAttrs}) должен быть пуст.
\end{enumerate}
\end{enumerate}

\section{Конвертованные данные}\label{FMT.EnvelopedData}

Формат конвертованных данных задается типом~\texttt{EnvelopedData}, который
определен в СТБ 34.101.23. 

Контейнер~\texttt{EnvelopedData} заполняется по правилам СТБ~34.101.23
со следующими уточнениями.

\begin{enumerate}
\item
Версия синтаксиса (компонент~\texttt{version}) должна равняться~$2$. 

\item
Опциональные компоненты~\texttt{originatorInfo} и 
\texttt{unprotectedAttrs} должны быть опущены. 

\item
Идентификатор алгоритмов шифрования 
(компонент~\texttt{contentEncryptionAlgorithm}, вложенный  
в~\texttt{encryptedContentInfo}) должен быть выбран из перечня, 
заданного в~\ref{CRYPTO.Encr}.

\item
Тип конвертуемых данных (компонент~\texttt{eContentType}, вложенный 
в~\texttt{encryptedContentInfo}) должен принимать одно из следующих значений:
\begin{enumerate}
\item[1)]
\texttt{id-signedData}, если конвертуются подписанные данные;
\item[2)]
\texttt{id-data}, если конвертуются неструктурированные данные:
запрос на получение сертификата, сертификат, запрос на отзыв сертификата.
\end{enumerate}

\item
Список сведений о получателях (компонент~\texttt{recipientInfos})
должен содержать единственный контейнер~\texttt{RecipientInfo}, 
и в этом контейнере должен быть выбран компонент~\texttt{ktri} 
типа~\texttt{KeyTransRecipientInfo}. 

Компонент~\texttt{ktri} должен формироваться следующим образом:
\begin{enumerate}
\item
версия применяемого синтаксиса (компонент~\texttt{version})
должна равняться~2;
\item
идентификатор получателя (\texttt{rid}) должен задаваться через выбор 
варианта~\texttt{issuerAndSerialNumber};
\item
в~\texttt{keyEncryptionAlgortithm} должен быть установлен идентификатор
алгоритмов~\texttt{bign-keytransport}, описанных в~\ref{CRYPTO.Transport}.
\end{enumerate}
\end{enumerate}